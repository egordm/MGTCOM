\chapter{Future Work}
While we have explored a multitude of topics, there is still a lot of room for further improvements and exploration.
Below we list a various of possible further research directions.

Our experiments have shown that temporal representation learning benefits greatly from auxiliary embeddings as node features may often be too weakly correlated with temporality.
In contrast to topological tasks, auxiliary embeddings have been shown to be effective only for the most important nodes.
In future work, it may be valuable to explore more flexible settings where representations are augmented with embeddings only for temporal tasks, therefore reducing parameters and inference latency.

The scale of our model is meta-topology bound, meaning that the amount of learnable parameters increases if there are more node or edge types.
This reduces the effectiveness of our framework on highly heterogeneous networks such as knowledge graphs.
Future works may explore improvements to our embedding method by utilizing techniques used in the knowledge graph embedding field.

While detected communities excel in topological and temporal predictive capabilities, they detected communities still under-perform on the modularity measure.
Further work may explore swapping node2vec random walk algorithm by motif-sampling \cite{jiaCommunityGANCommunityDetection2019} to encourage strong link-based proximity.

The presented framework uses DPMMSC algorithm as introduced in the original paper \cite{changParallelSamplingDP2013a}. 
Meanwhile, a multitude of works has been published that extend the algorithm to a deep learning setting \cite{ronenDeepDPMDeepClustering2022} or that address local minima issues faced by the algorithm.
Hierarchical DPMM algorithms have been studied \cite{tehHierarchicalDirichletProcesses2006, changSamplingComputerVision} and may be invaluable for community detection in analytical settings.
Our clustering implementation can be further improved by exploring the effectiveness of different priors and introducing new split/merge proposal methods.
Finally, we note that the detected communities are mainly dictated by the structure of node embeddings.
Introducing a control parameter to bias communities towards temporal and topological communities would improve ergonomics of community detection when reusing the learned embeddings.

\chapter{Conclusion}
In this paper, we introduce the MGTCOM framework for community detection in multimodal graphs.  It utilizes meta-topological, topological, content features, and temporal information to detect communities.
Moreover, we address common issues in multimodal graphs such as information incompleteness, and inference on unseen data by adopting a graph convolutional network architecture that combines k-hop neighborhood sampling and random walk context sampling.
We devise a unified objective and an efficient temporal sampling method to learn multimodal community-aware node embeddings in an unsupervised manner.
Consequently, we leverage a split/merge-based Dirichlet process mixture model for community detection where the number of communities are not known a priori.
Our empirical evaluation shows that MGTCOM is quite competitive with the state-of-the-art. 
